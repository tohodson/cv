\documentclass{cv_TOH}

\usepackage[letterpaper,top=1cm,bottom=1cm,left=2cm,right=2cm]{geometry}
\usepackage{hyperref}

\pagenumbering{gobble}

\begin{document}

\name{Timothy O. Hodson}
\shortcontact{tohodson@gmail.com}{(217)-493-5551}
 
\section{Education}
\datedsubsection{Ph.D. Student in Geology -- Northern Illinois University}{In progress}
\datedsubsection{Certificate of Graduate Study in Geographic Information Analysis}{2014}
\datedsubsection{M.S Geology -- Northern Illinois University}{2012}
\datedsubsection{B.S. Geology and Geophysics -- University of Wisconsin, Madison}{2007}

\section{Research Interests}
\begin{itemize}
\item Ice sheet stability
\item Application of remote sensing and geophysical data to surficial and seafloor mapping. 
\item Realtime data viz from distributed sensor networks
\end{itemize}
\section{Professional Experience}

\datedsubsection{Research Assistant, Northern Illinois University}{2010--Present}

\begin{itemize}
\item Participant in \href{http://www.wissard.org}{WISSARD}, a NSF funded project to explore aquatic and marine environments beneath Whillans Ice Stream and Ross Ice Shelf, Antarctica. Responsibilities include operations and maintenance of oceanographic instruments, data acquisition and post-processing software development, and planning operations and logistics.
\item Mapped seafloor habitats in Puget Sound, Washington by mining historical bathymetric surveys and seafloor observations. Resulting mapset are in review with USGS and are being used by the Nature Conservancy to inform the region's public and policy makers.
\item \href{http://pubs.usgs.gov/sim/3253/}{Mapped seafloor habitats in Glacier Bay National Park using remotely sensed data and seafloor video observations in a maximum likelihood classifier. Resulting report and mapset published by USGS.}
\end{itemize}

\subsubsection{Teaching Assistant, Northern Illinois University}
Teaching assistant for courses in introductory geology, sedimentology, vertebrate paleontology and planetary and space science.

\datedsubsection{Research Assistant, Illinois State Geological Survey}{2008--2009}

Supported geolgists in surfcial mapping, aquifer characterization, and watershed characterization.

\section{Other Experience}
\subsection{Mapping Intern -- NOAA}
\subsection{STEMfest Volunteer}

\section{Programming}
\begin{itemize}
\item Python, C, Matlab, R, BASH, JavaScript, SQL, HTML/CSS
\item Relevant coursework in geospatial methods and modeling, web mapping, and network applications in C
\item Familiarity with common libraries for numerical programming, image analysis, geospatial analysis, statistics and machine learning
\end{itemize}

\section{Software}
ArcGIS (4yrs), Adobe Illustrator, QGIS, GRASS, Fledermaus, Microsoft Office, Unix utilities, Emacs, and Vim

\section{References}
\begin{tabular}{@{}p{6cm}p{6cm}p{6cm}}
\textbf{Ross Powell}                  &  \textbf{Thomas Pingel}            & \textbf{John Winans}   \\
Professor                             &  Asst. Professor                   &  Research Associate    \\
Geology and Env. Geoscience           &  Geography                         &  Computer Science      \\
Northern Illinois University          &  Northern Illinois University      &  Northern Illinois University \\
DeKalb, Illinois                      &  DeKalb, Illinois                  &  DeKalb, Illinois  \\
phone: 815.753.7952                   &  phone: 815.753.9208               &  phone: 815.753.5947  \\
e-mail: rpowell@niu.edu               &  e-mail: tpingel@niu.edu           &  e-mail: jwinans@niu.edu   \\
                                      &  \url{www.tpingel.com}             &  \url{faculty.cs.niu.edu/~winans} \\
\end{tabular}

\end{document} 
