%% This is file `cvLorenaBarbaUS.tex',
%% LaTeX file using the currvita package
%%
%% Copyright (c) Lorena Barba


% -----> process with LaTeX (and not XeLaTeX)


\documentclass[10pt]{article}

\usepackage{bibunits}
\usepackage[ManyBibs]{currvita}
\usepackage{eurosym}


%\usepackage[light,nomath]{kpfonts}
\usepackage[light,fulloldstylenums,nomath]{kpfonts}  % Kp fonts, derived from URW Palladio font family


\usepackage{geometry}
\geometry{textwidth=6.1in, textheight=9.2in}  
% US letter size is 8.5 x 11 in

\usepackage{color}
% Define the color to use in links:
\definecolor{linkcol}{rgb}{0.459,0.071,0.294}
% Used DigitalColor Meter to determine, RGB equivalent 117, 18, 75

%\renewcommand*{\cvheadingfont}{\large\bfseries}  % Optional: change heading font
%\renewcommand*{\cvlistheadingfont}{\scshape}	  % Optional: change cvlist heading font
%\renewcommand*{\cvlabelfont}{\itshape}			% Optional: change labels of cvlist font
\setlength{\cvlabelwidth}{70mm}

% hyperref package should always be loaded as the very last one
% to be sure that it has the last word ...
\usepackage[
    pdftex,
    pdftitle={Lorena Barba's Curriculum Vitae},
    pdfauthor={LorenaBarba},
    pdfpagemode={UseOutlines},
    bookmarks, bookmarksopen,bookmarksnumbered={True},
    colorlinks, linkcolor={linkcol},citecolor={linkcol},urlcolor={linkcol}
    ]{hyperref}

%\newcites{journal}{Academic journal papers, refereed}
%\newcites{confref}{Conference papers, refereed}
%\newcites{conf}{Conference contributions}
%\newcites{other}{Other publications}


\newcommand*{\ac}[1]{\mbox{#1}}

 \tolerance=600

\begin{document}
%\bibliographystyle{plain}
%\bibliographystylejournal{unsrt}
%\bibliographystyleconfref{unsrt}
%\bibliographystyleconf{unsrt}
%\bibliographystyleother{unsrt}

\begin{cv}{Curriculum Vitae}
\hrule

\medskip

%%%%%%%%%%%%%%%%%%%%%%%%%%%%%%%%%%%%%%%%%%%%%%%%%%%%%%%%%%%%%%%%%%%%%%%%%%

  \begin{cvlist}{}


  \item 
  \begin{minipage}[t]{2.2in}
  	Lorena A. Barba, PhD\\   Washington, DC \\ cell phone:~(617)~909-5900 \\ office phone:~ (202)~994-3715\\
  \end{minipage} \hfill
  \begin{minipage}[t]{3in}  
    	email: \href{mailto:labarba@gwu.edu}{labarba@gwu.edu}\\
    	website: \href{http://lorenabarba.com/}{http://lorenabarba.com}\\
	Twitter \href{https://twitter.com/LorenaABarba}{@LorenaABarba}
   \end{minipage} 
  \item    Nationality: Chile, Croatia. US Permanent Resident.
  \end{cvlist}



%%%%%%%%%%%%%%%%%%%%%%%%%%%%%%%%%%%%%%%%%%%%

  \begin{cvlist}{Current Position}


  \item[8/2013--] Associate Professor, Mechanical and Aerospace Engineering (with tenure from 9/2014).   The George Washington University, Washington, DC.
  \end{cvlist}


%%%%%%%%%%%%%%%%%%%%%%%%%%%%%%%%%%%%%%%%%%%%

  \begin{cvlist}{Previous Academic Positions}
  \item[1/2016--5/2016] Visiting Scholar, Berkeley Institute of Data Science (BIDS), University of California Berkeley.

  \item[9/2008--7/2013] Assistant Professor (tenure-track), Mechanical Engineering, Boston University.

  \item[8/2004--9/2008] Lecturer in Applied Mathematics (permanent position from 8/2007), University of Bristol, UK. Promoted to Senior Lecturer while on leave, 2009.

 % \item[9/1998--6/2001] Graduate Research Assistant with Prof.~Anthony Leonard,   Graduate Aeronautical Laboratories, \textsc{galcit}, California Institute of Technology, Pasadena CA

   \item[3/1996--6/1998] Adjunct professor, Department of Environmental Engineering, 
  Universidad de Vi\~na del Mar, Vi\~na del Mar, Chile.

  \end{cvlist}

%%%%%%%%%%%%%%%%%%%%%%%%%%%%%%%%%%%%%%%%%%%%%%%%%%%%%%

  \begin{cvlist}{Previous Industrial Position}

%%%%%%%%%%%%%%%%%%%%%%%%%%%%%%%%%%%%%%%%%%%%%%%%%%%%%%

  \item[1991--1998] Consulting engineer and partner, 
  \textsl{Consultora en Ingenier\'ia Mec\'anica Gear Ltd.} Valpara\'iso, Chile.

  \end{cvlist}
%%%%%%%%%%%%%%%%%%%%%%%%%%%%%%%%%%%%%%%%%%%%%%%%%%%%%%

  \begin{cvlist}{Academic Qualifications}

%%%%%%%%%%%%%%%%%%%%%%%%%%%%%%%%%%%%%%%%%%%%%%%%%%%%%%

  \item[6/2004] \textbf{PhD}, Aeronautics\\
  California Institute of Technology, Pasadena CA, USA.\\
  Thesis: \textsl{Vortex method for computing high-Reynolds number flows:
  Increased accuracy with a fully mesh-less formulation.}
  Supervisor: Prof.~Anthony Leonard.

  \item[6/1999] \textbf{MSc}, Aeronautics\\
  California Institute of Technology, Pasadena CA, USA.

  \item[8/1998] \textit{Ingeniero Civil Mec\'anico} (\textbf{Professional Engineer})\\
  Universidad T\'ecnica Federico Santa Mar\'ia, Chile\\
  Thesis: \textsl{An\'alisis de sistemas de aeraci\'on para tratamiento de
  aguas residuales:  Bases te\'oricas, dise\~no y evaluaci\'on de
  aireadores.}
  Advisor: Prof.~Pedro Roth.

  \item[6/1989] \textbf{BSc, First Class honours}, Mechanical Engineering\\
  Universidad T\'ecnica Federico Santa Mar\'ia,  Valpara\'iso, Chile.
  



  \item[\textbf{Non-degree graduate studies:}]
  \item[1991-1992] Renewable Energies, MSc level courses, part-time study at 
  Universidad T\'ecnica Federico Santa Mar\'ia, Valpara\'iso, Chile.\\

  \end{cvlist}



%%%%%%%%%%%%%%%%%%%%%%%%%%%%%%%%%%%%%%%%%%%%%%%%%%%%%%%%%%%%%%%

  \begin{cvlist}{Awards, Honours \& Distinctions}

%%%%%%%%%%%%%%%%%%%%%%%%%%%%%%%%%%%%%%%%%%%%%%%%%%%%%%%%%%%%%%
  \item[Aug.~2012] Selected for the NAE Frontiers of Engineering Education Symposium (FOEE), of the National Academy of Engineering.
  \item[Apr.~2012] Appointed CUDA Fellow, NVIDIA Corp.
  \item[Feb.~2012] National Science Foundation CAREER award.
  \item[Aug.~2011] NVIDIA Academic Partnership award (\textdollaroldstyle 25,000 unrestricted cash award).
  
  \item[Jan.~2008] \textit{Rising Star} Teaching Award for the Faculty of Science, University of Bristol.
  
  \item[3/2005--3/2007]  The Nuffield   Foundation Award to Newly Appointed Lecturers in Science, Engineering and Mathematics (\pounds 5,000).

  \item[9/1998--5/2001]  Graduate Research Fellowship, Graduate
  Aeronautical Laboratories, California Institute of Technology,
  Pasadena.


  \item[1999--2000] \textit{Amelia Earhart Fellowship Award}, for aerospace-related
  graduate studies at doctoral level, Zonta International Foundation (\textdollaroldstyle 6000).

  \item[1994] Invited by the Air Force of Chile to be part of the
  first Reserve Officers course for women pilots (only 13 women in the country were invited).  Received the rank of Ensign (March 1995). Promoted to the rank of 2nd Lieutenant (Reserve), May 2000.

  \item[1989] Ranked 1st of the graduating class,
  best \textsc{gpa} over the 12-semester program (completed in 10 semesters); 2nd place in admission scores of the university (first to the Mechanical Engineering program), Universidad T\'ecnica F. Santa Mar\'ia.

  \end{cvlist}


   \hrule

  
  %%%%%%%%%%%%%%%%%%%%%%%%%%%%%%%%%%%%%%%%%%%%%%%%%%%%%%%%%%%%%%%%%%%%%%%%%%
  \begin{cvlist}{Research }

   % \medskip

  %%%%%%%%%%%%%%%%%%%%%%%%%%%%%%%%%%%%%%%%%%%%%%%%%%%%%%%%%%%%%%%%%%%%%%%%%%

  \item[\textit{Keywords:}] Computational science, fluid dynamics, animal flight, computational biophysics, immersed boundary method, particle methods, radial basis functions, parallel computing, fast multipole methods, boundary element methods, many-core hardware, \textsc{gpu} computing.

 \item[Google scholar:] $h$-index$=14$, \href{http://scholar.google.com/citations?user=djTaZ7UAAAAJ}{http://scholar.google.com/citations?user=djTaZ7UAAAAJ}
 \item[Impact Story profile:] \href{https://impactstory.org/lorenabarba}{https://impactstory.org/lorenabarba}
  \item[Figshare profile:] \href{https://figshare.com/authors/Lorena\_A\_Barba/97553}{https://figshare.com/authors/Lorena\_A\_Barba/97553}
  \end{cvlist}

%  PUBLICATIONS                %%%%%%%%%%%%%%%%%%%%%%%%%%%
%  PUBLICATIONS                %%%%%%%%%%%%%%%%%%%%%%%%%%%

\begin{cvlist}{Publications}
  
%  \item[]  \textit{The star symbol $\star$ indicates that the co-author is/was my graduate student or postdoc, and for conference contributions the student may have given the talk.}
%%%%%%%%%%%%%%%%%%%%%%%%%%%%%%%%%%%%%%%%%%%
% To process bibliography:
% latex cvLorenaBarbaUS.tex
% bibtex bu1
% bibtex bu2
% bibtex bu3
% bibtex bu4
% latex cvLorenaBarbaUS.tex

%   SUBMITTED / PREPRINTS
    \begin{bibunit}[unsrt]
    \nocite{MesnardBarba2016,LaytonWangBarba2016}
      \item[\textbf{Submitted or in preprint}]
    \item[] \putbib[barba]
    \end{bibunit}


%  PUBLISHED JOURNAL PAPERS
    \begin{bibunit}[unsrt]
    \nocite{}
    \item[\textbf{Journal papers}]
    \item[] \putbib[barba]
    \end{bibunit}

% TECHNICAL REPORT
    \begin{bibunit}[unsrt]
    \nocite{}
    \item[\textbf{Technical Reports}]
    \item[] \putbib[barba]
    \end{bibunit}
  \end{cvlist}

% CONFERENCE CONTRIBUTIONS
   \begin{bibunit}[unsrt]
   \nocite{} 
   \item[\textbf{Conference contributions, with talk (unless otherwise noted)}]
    \item[] \putbib[barba]
    \end{bibunit}

\end{cvlist}

%%%%%%%%%%%%%%%%%%%%%%%%%%%%%%%%%%%%%%%%%%%
% To process bibliography:
% latex cvLorenaBarbaUS.tex
% bibtex bu1
% bibtex bu2
% bibtex bu3
% bibtex bu4 ...
% latex cvLorenaBarbaUS.tex






\begin{cvlist}{Research Codes}

\item[] \textit{Our reproducibility policy is to open all research code and release at the time of manuscript submission. The following are the most recently available codes.}

\item[\texttt{exaFMM}]  An open-source code base for fast multipole algorithms, in parallel and with GPU capability. The fast multipole method (FMM) is a numerical engine used in many applications including acoustics, electrostatics, fluid simulations, wave scattering and more. See \href{http://exafmm.org}{http://exafmm.org} (2011).
\item[\texttt{PyGBe}] Solves the linearized Poisson-Boltzmann equation using a boundary-integral formulation (boundary element method, BEM). The application is electrostatics of biological macro-molecules. Under development by PhD student Christopher Cooper. See \href{https://github.com/barbagroup/pygbe}{https://github.com/barbagroup/pygbe} (2012).
\item[\texttt{cuIBM}] Navier-Stokes solver using immersed boundary methods on GPU hardware with CUDA. Under development by PhD student Anush Krishnan. \\ See \href{https://github.com/barbagroup/cuIBM}{https://github.com/barbagroup/cuIBM} (2012).

\end{cvlist}

%  GRANTS                %%%%%%%%%%%%%%%%%%%%%%%%%%%

  \begin{cvlist}{Grants}
  
 \item[\textit{Principal Investigator, unless otherwise noted.}]  
 %%%%%%%%%%%%%%%%%%%%%%%%%%% %%%%%%%%%%%%%%

\item[6/2014--12/2014]~``A multi-campus, blended, connected course and MOOC,'' with support from the GW Office of the Vice-Provost for Online Learning and Academic Innovation (\textdollar 15,000), School of Engineering and Applied Sciences (\textdollar 20,000); a cash donation by Nvidia, Inc. (\textdollar 12,000) and cloud-hosting credits awarded by an Amazon AWS Educational Grant (\textdollar 5,000). Total: \textdollar 52,000.

 \item[1/2014--6/2014] (\emph{co-PI} 1/2)~``Just-in-time Interactive Online Modules for Applied Engineering Computing,'' Grants for High-Impact Teaching and Learning Practices,  The Office of Teaching \& Learning, George Washington University. With Adam M. Wickenheiser; amount \textdollar 10,000 (with additional \textdollar 3,000 from the School of Engineering and Applied Science).
 
 \item[9/2013--9/2015]  (\emph{co-PI} 1/18)~``WIDER Planning: GRASP, GW Reform and Advancement of STEM-education Practices,'' NSF Division of Undergraduate Education, award \#DUE-1347516, amount \textdollar 229,886. Principal Investigator: Rahul Simha, Co-PIs: Stephen Ehrmann, with 16 other GW faculty, including Barba.

 \item[9/2012--8/2013] ``Pan American Advanced Studies Institute: The Science of Predicting and Understanding Tsunamis, Storm Surges, and Tidal Phenomena,'' NSF Office of International Science and Engineering, award \#OISE-1242245, amount \textdollar 99,945. 
 
 \item[2/2012--1/2017] ``CAREER:  Scalable Algorithms for Extreme Computing on Heterogeneous Hardware, with Applications in Fluids and Biology'', NSF Office of Cyberinfrastructure, award \#OCI-1149784, amount \textdollar 550,627.
 
 \item[1/2012--12/2012] ``Tree- and multipole-based algorithms for exascale computing,'' Massachusetts Green High-Performance Computing Center (MGHPCC) seed grant; co-PIs: Cris Cecka (Harvard University) and Hans Johnston (Univ.\ of Massachusetts, Amherst); amount \textdollar 130,487 (BU portion \textdollar 40,132).
 
 \item[8/2011--8/2012] ``Computational fluid dynamics with immersed boundary method on GPU hardware,'' NVIDIA Academic Partnership Program (APP), \textdollar 25,000.
 
 \item[9/2011--8/2012] ``Post-PASI Workshop: Solidifying networks and staying current in parallel computing,'' NSF Office of International Science and Engineering, award \#OISE-1143988, amount \textdollar 31,713. 
 
 \item[2/2011--1/2014] ``Multiplying algorithmic and hardware speed-up:  fast multipole boundary element method on GPU, applied to biomolecular electrostatics,'' Office of Naval Research (ONR) award \#N00014-11-1-0356, amount \textdollar 159,318.
 
 \item[9/2010--8/2011] ``Pan-American Advanced Studies Institute (PASI)---Scientific Computing in the Americas: the challenge of massive parallelism,'' NSF Office of International Science and Engineering, award \#OISE-1036435, amount \textdollar 100,600.  \\ Supplementary award \#OISE-1104259, amount \textdollar 15,400. \\ See: \href{http://www.bu.edu/pasi/}{www.bu.edu/pasi/}
 
 \item[01/2011--4/2011] WIN Mary Erskine Faculty Grant Type I (\textdollar 1000) and Type 2 (\textdollar 500), Boston University NSF ADVANCE program WIN: Women in Networks.
 
 \item[9/2010--8/2011] Teragrid start-up allocation, ``Scalable Hierarchical Algorithms with Applications in Fluids and Biology,'' NSF Teragrid award \#TG-ASC100042,  amount 100,000 SUs.  
 
 \item[9/2009--8/2011]  (\emph{co-PI} 1/3)~``Experimental GPU cluster for fundamental physics,'' NSF Office of Cyber Infrastructure, award \#OCI-0946441, amount \textdollar 297,971.  With PI Richard Brower and co-PI Claudio Rebbi.
 
  \item[7/2007--6/2010]  ``Development of modular and scalable hybrid numerical algorithms for flow simulation, combining continuum approaches with particle methods'' BAE Systems \& Airbus UK, Centre for Fluid Mechanics Simulation (CFMS), amount \pounds 85,640. 
  
  \item[7/2007--6/2010] ``Numerical study of high-Reynolds number flow with high-order accurate meshless vortex method,'' Engineering and Physical Sciences Research Council (EPSRC), First Grant Scheme, Ref.\ \href{http://gow.epsrc.ac.uk/ViewGrant.aspx?GrantRef=EP/E033083/1}{EP/E033083/1}, amount \pounds 209,238.

   \item[3/2006--3/2007] British Council/NWO Partnership Programme in  Science, Ref.\ \textsc{pps} 848, amount \pounds 1300 for collaboration with Eindhoven University of Technology, Fluid Dynamics Lab., Dept. of Applied Physics.

   \item[11/2005--7/2009] Programme ALFA II, Project ``SCAT---Scientific Computing Advanced Training'' Ref. II-0537-FC-FA, European Commission, EuropeAid Co-operation Office.  Collaboration project with a total budget of \euro 1.34 million, involving 10 institutions in 6 countries of Europe and Latin America (with Barba as sole P.I.)

   \item[3/2005--3/2007] Award to Newly Appointed Lecturers in Science, Engineering and Mathematics 2005, Nuffield Foundation, Ref. NAL/01164/G, amount \pounds 5,000 .

  \end{cvlist}


\medskip

 % Indications of external recognition
\input{recognition}
 
 % Professional service
 \input{service}

\hrule



 % TEACHING                 %%%%%%%%%%%%%%%%%%%%%%%%%%
  \begin{cvlist}{Teaching}
    %\hrule
    \medskip
  %%%%%%%%%%%%%%%%%%%%%%%%%%%%%%%%%%%%%%

%\item[Teaching Portfolio (2007) online at:]
%\href{http://www.maths.bris.ac.uk/~aelab/teaching/}{http://www.maths.bris.ac.uk/$\sim$aelab/teaching/}


 \item[\textbf{Awards \& Honors}] Selected for the NAE Frontiers of Engineering Education (FOEE) Symposium, National Academy of Engineering, 2012\,---\,``brings together some of the nation�s most engaged and innovative engineering educators in order to recognize, reward, and promote effective, substantive, and inspirational engineering education.'' See \href{http://www.naefoee.org}{http://www.naefoee.org}
 
 \item[] \href{http://www.bristol.ac.uk/esu/academicdevelopment/prizes/previouswinners.html#2007}{Rising Star} Teaching Award, Univ.\ of Bristol, 2008\,---\,aimed
 at a faculty member who has been teaching in Higher
 Ed.\ for less than five years, it recognizes the quality of 
 reflection, analysis and practice in the individual's teaching.

        \item[\textbf{Teaching technology}]  Experience with lecture capture, production of podcasts and screencasts.  Experience with \href{http://www.apple.com/education/itunes-u/}{iTunes U} and YouTube for public dissemination of course content. Multiple open-course-ware offerings, including the CFD lectures on You Tube, where they have amassed over 300,000 views. User of web-based student-response system \textsl{Socrative} and social-learning platform \textsl{Piazza}. Expert user of the \textsl{Open edX} course platform for massive open online courses (MOOCs).
  
  
  
 
\item[\textbf{Experience}] 
  
 \item[\emph{George Washington University,}]
 
 \item[Fall'14 \& '15] Numerical Methods in Mechanical and Aerospace Engineering (MAE6286), first-year graduate students  (or advanced elective for seniors).
 
 \item[Spring'14 \& '15] Aero/Hydrodynamics (MAE6226), first-year graduate students  (or advanced elective for seniors).
 
\item[Fall'13] Bio-aerial Locomotion, as a Special Topics course (MAE6221), first-year graduate students  (or advanced elective for seniors).
 
 \item[\emph{Boston University,}]
 
 \item[Fall'11 \& '12] Bio-aerial Locomotion (EK131/132), freshman undergraduate, with \href{http://www.nvidia.com/object/sc11.html}{course blog} and media \href{http://itunes.apple.com/us/itunes-u/bio-aerial-locomotion-ek131/id464937253}{on iTunes U}.
 
   \item[Spring'11, '12 \& '13]  Computational Fluid Dynamics (ME702), graduate course.
 
 \item[Fall'10] Fluid Mechanics, ME303 (now on \href{http://itunes.apple.com/us/itunes-u/fluid-mechanics-2010-eng-me303/id452560560}{iTunes U})
  \item[Spring'10]  Computational Fluid Dynamics (ME702), graduate course, with lectures on \href{http://itunes.apple.com/itunes-u/computational-fluid-dynamics/id452560554}{iTunes U}
  \item[Fall'09 \& Spring'10] Undergraduate Fluid Mechanics (ME303).
  
  \medskip
  
    \item[\emph{University of Bristol, UK}]
    \item[2004--8] Four years of teaching experience in the undergraduate
  program of the Dept.\ of Mathematics, with consistent success as
  evidenced in my Teaching Portfolio.    Courses:  Ordinary Differential Equations, Introduction to Physical Modeling.
 
    
 \bigskip 
    
   \item[\textbf{Student advising}]
   
   \item[PhD students, new at GW:] 
      Gilbert Forsyth, Natalia Clementi, Pi-Yueh Chuang, Tingu Wang.
       Olivier Mesnard (started Sept.~2012 at BU). 

      
   \item[PhDs graduated:]  At Boston University---
   \item Christopher V. Cooper (2015); assistant professor at Universidad T\'ecnica Federico Santa Mar\'ia, Valparaiso, Chile.
   \item Anush Krishnan (2015); CFD Engineer at Exa Corp., Massachusetts. 
   \item[] Simon K. Layton (2013); DevTech Engineer at Nvidia Corp., California. 
   \item[] At University of Bristol, U.K.---Felipe A. Cruz (2011), Applied Mathematics. Followed by post-doc at the Nagasaki Advanced Computing Center, Japan; next, HPC Engineer at Square Enix (video game developer and publisher, makers of Final Fantasy, Dragon Quest and Tomb Raider), at El Segundo, CA.
  %``Development of modular and scalable hybrid numerical algorithms for flow simulation, combining continuum approaches with particle methods'',
   
   \item[MSc students]   Jonathan Brent Parham, 2012--13, Boston University. At University of Bristol, visiting students, grantees of the SCAT project between 2006 and 2008: Christopher V. Cooper, Paola Arce, Guillermo Cabrera, Claudio Torres, Felipe A. Cruz.
   
   \item[Undergraduate students] Alejandro Pelaez (BU Eng), UROP grantee Summer 2011. \\
      Olivier Mesnard, visiting from Institut Sup{\'e}rieur de M{\'e}canique de Paris, SUPMECA, Paris (Apr.--Aug. 2010).
      Simon K. Layton, University of Bristol (Oct.--Jun. 2008).


\item[\textbf{External examiner}]

 \item[PhD committee:] Aparna Chandramowlishwaran, Computational Science and Engineering Department, Georgia Institute of Technology (2012).
 
 \item[PhD thesis written evaluation:] Robert Speck, Dept.~Mathematics and Science at Wuppertal University, Germany (2011).






\item[\textbf{Courses developed}]

\item[Fall 2014] ``Practical Numerical Methods with Python''\,---\,developed as a massive open online course (MOOC), led simultaneously with the on-campus course, and developed a full set of open materials, publicly available as \href{http://lorenabarba.com/news/announcing-practical-numerical-methods-with-python-mooc/}{``NumericalMOOC''}.

\item[Spring 2014] Aero/Hydrodynamics\,---\,developed a set of lectures in potential flow and development of panel methods, using IPython Notebooks. Publicly available as \href{http://lorenabarba.com/blog/announcing-aeropython/}{``AeroPython.''}

\item[Fall 2013] Bio-aerial Locomotion\,---\,re-developed as a first-year graduate course, now drawing from the latest research in animal flight and with added focus on critical-thinking, research and writing skills. 

\item[Fall 2011] Bio-aerial Locomotion\,---\,freshman Introduction to Engineering module, completely new concept and design, emphasizing bio-inspired engineering.

\item[Spring 2010] Computational Fluid Dynamics\,---\,graduate course, fully designed by me. This course is currently available for free to the public in \href{http://www.apple.com/education/mobile-learning/}{iTunes U}.

\item[Spring 2009] Fluid Mechanics---junior engineering course. Lectures from the 2010 version of their course are available on \href{https://itunes.apple.com/itunes-u/fluid-mechanics-2010-eng-me303/id452560560}{iTunes U}.

\item[2007--8] Introduction to Physical Modeling\,---\,a completely new course for first year undergraduates (enrollment: 177), introduced as part of the recommendations of a Curriculum Review.

\item[2004--7] Ordinary Differential Equations, level 2 undergraduate course.



%  \item[\textbf{External examiner comments}]
%Quote from the 2005 report of the applied mathematics external examiner of
%  the University of Bristol;  I was the instructor of ODEs 2:
%  \item[]    \textit{``The Applied Mathematics courses within the Mathematics
%    Department are of a uniformly high standard. Several courses were
%    of a particularly high standard notably [course X], ODEs 2 and
%    [course Y].''}  \\




   \item[\textbf{Teaching and Learning in Higher Education Programme, U. of Bristol}]
   \item[Completed 6/2007]   Assessment of the programme Director, R. Brawn (R.Brawn@bristol.ac.uk): 
   \item[] \textit{``This is a first class portfolio ---and an exemplar of what we seek to do via the TLHE programme ...  It's clear from your reflections and evidence that you are a committed, enthusiastic and competent lecturer, who has student learning and enjoyment of their studies at heart ... Your commitment to core professional values is evident.''  ``Your writing shows clear evidence of your understanding of some key ideas in unit design.'' ``You show you care deeply about supporting learners within a context that recognizes a differentiated approach whilst maintaining a realistic outlook that acknowledges normal working contexts, cultures and pressures.''  }




  \item[\textbf{Previous experience}]
  \item[2000--01] Teaching
  assistant for the graduate course on Fluid Mechanics (PhD level), CALTECH, including weekly tutorial sessions with a section of the students, homework grading, and assisting in
  grading the examinations.
  
\item[1998] Thesis supervisor for two students of the
  Environmental Eng.\ program at Universidad de
  Vi\~na del Mar, Chile.  One received
  the honor of the Thesis Award of the Chamber of Deputies  (lower house of the National Congress) in 1999.

  \item[1996--97] Taught two senior-year courses
  for the Dept.\ of Environmental Eng., Universidad de
  Vi\~na del Mar, Chile:  Environmental Impact Assessment I \& II.  These courses were fully
  designed by me.

\item[1986--89] Teaching assistant at
  Universidad T\'ecnica F. Santa Mar\'ia (weekly contact sessions or problems classes and grading
  of homework and exams) for:  Calculus 1, Algebra 2, Physics 3, Calculus 4,
  Physics 130, Probability and Statistics.

  \end{cvlist}

 %%%%%%%%%%%%%%%%%%%%%%%%%%%
  \begin{cvlist}{Academic administration}
      \hrule
    \medskip
%%%%%%%%%%%%%%%%%%%%%%%%%%%%%%%%%%%%%



\item[9/2014--4/2015] Working Group on Participation, in service of the Sub-Committee on Faculty Governance, Committee on Academic Affairs, GW Board of Trustees.

\item[9/2013--] SEAS Computing Committee, School of Engineering and Applied Science, GW

\item[9/2013--9/2014] University Strategic Planning Committee on Online Education, GW.

\item[10/2012--5/2013] Engineering IT Advisory Committee, providing guidance and recommendations for the college of Engineering on the use of information technology in education and research; Boston University.

\item[9/2012--5/2013] Member, Boston University Teaching \& Learning Technologies Governance Committee.

\item[9/2009--5/2013] Member, Thermofluids Committee. 

\item[9/2008--5/2013]  Member, Department Website Committee.  Regular meetings and team work for the modernization of the website.

\item[Fall 2008]  Member, Graduate Committee.  Weekly meetings and evaluation of the graduate applications received.
  


\item[In University of Bristol, Department of Mathematics,]
 
  \item[11/2006--9/2008] Direct maintenance/feature
  enhancement of departmental website as Managing
  Editor.
  
  \item[11/2005--10/2006] Project manager, new department website development.  Interacted with an external design company, a journalist and copy writer, a photographer, and programmers.  The website
  incorporated dynamic pages and databases for people,
  publications, funded projects and events, as well as faculty
  profiles and several full-length research highlights.
  
  
\item[11/2005--11/2008]
Co-ordination and leadership of the SCAT project (ALFA Programme, EC).  This included the management of one employee, hired exclusively to work
on this project as project manager and financial officer.

  \item[8/2004--8/2007]
  Enterprise Leader for the Faculty of Science, University of Bristol: Promote and facilitate enterprise at the faculty/departmental
  level, encourage/mentor colleagues in identifying commercial potential of their  work. Attended course on patents, by the law firm   Withers \& Rogers on Sept. 2004.




 \item[\textbf{Professional memberships}]
  \item[]
	American Institute of Aeronautics and Astronautics, AIAA\\
	American Physical Society, APS\\
	Association for Computing Machinery, ACM\\
	Society for Industrial and Applied Mathematics, SIAM\\
	US Association for Computational Mechanics, USACM


  \end{cvlist}

\noindent To see the latest version of this document, go to:\\
\href{http://dl.dropbox.com/u/59763/cvLorenaBarbaUS.pdf}{http://dl.dropbox.com/u/59763/cvLorenaBarbaUS.pdf}


 \cvplace{Washington, DC}
 \date{\today}

\end{cv}

\end{document}

\endinput



%% End of file `cvLorenaBarba.tex'.
